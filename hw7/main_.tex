\documentclass{article}

\usepackage{amsmath}
\usepackage{xcolor}

\usepackage[T2A]{fontenc}
\usepackage[utf8]{inputenc}
\usepackage[russian]{babel}
\documentclass{standalone}
\usepackage[dvipsnames]{xcolor}
\usepackage{tikz}
\usepackage{enumitem}
\usepackage{marvosym}
\usepackage{wasysym}
\usepackage{fontawesome5}
\usepackage{geometry}
\geometry{
	left=20mm,
	right=15mm,
	top=25mm,
	bottom=30mm
}

\begin{document}
	\thispagestyle{empty}
\fontsize{14}{16}\selectfont
\begin{center}
	\textbf{Домашняя работа №7 по курсу \TeX'a}
\end{center}
\hspace{20px}
\begin{center}
	Морозов Данила Егорович
\end{center}
\hspace{20px}
\begin{center}
	22 февраля 2024 г.
\end{center}
\hspace{100px}

\begin{center}
	\color{black}
	\addtocontents{toc}{\protect\color{blue}}
	\tableofcontents
\end{center}

	\newpage
\section{Неравенства Йенсена}
\fontsize{10}{20}
\textbf{Theorem 1.1}(Неравенства Йенсена).\textit{Пусть f(x) выпукла верх на }[a,b]\textit{.Тогда $\forall x_1,....,x_n$ $\in$ }[a,b]\textit{и их выпуклой комбинации выполнено неравнество $\sum_{k -= 1}^n$ $\alpha_k$ f(x$_k$) $\leq$ f($\sum_{k = 1}^n$ $\alpha_k$x$_k$)}\newline\newline
\textit{Доказательство.} (Докажем по индукции)\newline\newline
\textbf{База:} \textit{n} = 2
\begin{quote}
Неравенство превращается в определение выпуклой вверх функции,для которой это,очевидно,выполняется.
\end{quote}
\textbf{Переход:} Пусть это выполняется для \textit{n}.Докажем,что это работает и для \textit{n} + 1\newline
\[\sum_{k = 1}^{n + 1}\alpha_k = 1,\text{обозначим за} ~ s_n = \sum_{k = 1}^{n + 1}\alpha_k\]
\begin{quote}
Пусть $\beta_k$ = $\frac{\alpha_k}{s_n}$.Тогда получаем:$\sum_{k = 1}^n$$\beta_k$ = 1\newline
\[\mspace{-220mu}\sum_{k = 1}^{n + 1}\alpha_{k}f(x_k) = s_n\sum_{k = 1}^{n}\beta_k f(x_k) + \alpha_{n + 1}f(x_{n + 1}) \leq\]
\[\mspace{110mu}\leq \text{(по предположению индукции)} s_n\left( \sum_{k = 1}^n \beta_k x_k \right) + \alpha_{n + 1} f(x_n + 1) \leq\]
\[\mspace{550mu}\leq (\text{так как} s_n + \alpha_{n + 1} = 1) f\left(\sum_{k = 1}^{n + 1} \alpha_k x_k\right)\]
\end{quote}
Значит, шаг индукции проделан, неравенство доказано для произвольного n.
\[\mspace{850mu}\Square\]
\section{Круги Эйлера}
\fontsize{12}{10}
\begin{center}
	\color{red}$(A \cup B) \setminus (A \cap B)$
\end{center}
\begin{center}
\begin{tikzpicture}[thick,
	set/.style = {circle,
		minimum size = 3cm,
		fill= pink}]

	\node[set] (A) at (0,0) {};
	\node[set] (B) at (2,0) {};
	
	\begin{scope}
		\clip (0,0) circle(1.5cm);
		\clip (2,0) circle(1.5cm);
		\fill[white] (0,0) circle(1.5cm);
	\end{scope}

	\draw[white] (0,0) circle(1.5cm);
	\draw[white] (2,0) circle(1.5cm);	
	\node at (-0.5,0) {A};
	\node at (2.3,0) {B};
	\node at (1.03,0) {$A\Delta B$};
	
\end{tikzpicture}
\end{center}
\begin{center}
Рис. 1: Симметрическая разность
\end{center}
\end{document}
