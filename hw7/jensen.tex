\newtheorem{theorem}{Theorem}[section]

\section{Неравенства Йенсена}
\fontsize{10}{20}
\begin{theorem}
\textbf(Неравенства Йенсена).\textit{Пусть f(x) выпукла вверх на }[a,b]\textit{. Тогда  $\forall x_1,....,x_n$ $\in$ }[a,b]\textit{ и их выпуклой комбинации выполняется неравенство $\sum_{k = 1}^n$ $\alpha_k$ f(x$k$) $\leq$ f($\sum{k = 1}^n$ $\alpha_k$x$_k$)}\newline

\item\textbf{База:} \textit{n} = 2
\begin{quote}
Неравенство превращается в определение выпуклой вверх функции,для которой это,очевидно,выполняется.
\end{quote}
\item\textbf{Переход:} Пусть это выполняется для \textit{n}.Докажем,что это работает и для \textit{n} + 1\newline
\[\sum_{k = 1}^{n + 1}\alpha_k = 1,\text{обозначим за} ~ s_n = \sum_{k = 1}^{n + 1}\alpha_k\]
\begin{quote}
Пусть $\beta_k$ = $\frac{\alpha_k}{s_n}$.Тогда получаем:$\sum_{k = 1}^n$$\beta_k$ = 1\newline
\[\mspace{-220mu}\sum_{k = 1}^{n + 1}\alpha_{k}f(x_k) = s_n\sum_{k = 1}^{n}\beta_k f(x_k) + \alpha_{n + 1}f(x_{n + 1}) \leq\]
\[\mspace{110mu}\leq \text{(по предположению индукции)} s_n\left( \sum_{k = 1}^n \beta_k x_k \right) + \alpha_{n + 1} f(x_n + 1) \leq\]
\[\mspace{550mu}\leq (\text{так как} s_n + \alpha_{n + 1} = 1) f\left(\sum_{k = 1}^{n + 1} \alpha_k x_k\right)\]
\end{quote}
\begin{proof}
Значит, шаг индукции проделан, неравенство доказано для произвольного n.
\end{proof}
\end{theorem}
